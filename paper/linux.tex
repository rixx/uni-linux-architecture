% This is "sig-alternate.tex" V2.1 April 2013
% This file should be compiled with V2.5 of "sig-alternate.cls" May 2012

\documentclass{sig-alternate-05-2015}

% remove Copyright
\makeatletter
\def\@copyrightspace{\relax}
\makeatother

\begin{document}

\title{Requirements Engineering and Software Architecture in the Linux Kernel}
\subtitle{[Paper for the Requirements Engineering and Software Architecture course at Universität Stuttgart 2015/16]}

\numberofauthors{3}
\author{
    \alignauthor Julian Kuhn\\
        \affaddr{Universität Stuttgart}\\
        \email{mail@example.com}
    \alignauthor Friederike Kunze\\
        \affaddr{Universität Stuttgart}\\
        \email{mail@example.com}
    \alignauthor Oliver R{\"o}hrdanz\\
        \affaddr{Universität Stuttgart}\\
        \email{mail@example.com}
}

\maketitle

\begin{abstract}

This paper tries to detail both the requirements engineering process and the general software architecture of the Linux kernel project.

With regards to the requirements engineering, the paper dicusses the unusual workflow and its results within the development community.
It then continues to contrast this workflow with processes found with external contributors with their own interests in the Linux kernel development and maintenance.

TODO: welche spezifischen Architektur-Teile?

TODO: Zusammenfassung der Resultate

\end{abstract}

% The code below should be generated by the tool at
% http://dl.acm.org/ccs.cfm
% Please copy and paste the code instead of the example below.
%
\begin{CCSXML}
\end{CCSXML}

\ccsdesc[500]{Computer systems organization~Embedded systems}
\ccsdesc[300]{Computer systems organization~Redundancy}
\ccsdesc{Computer systems organization~Robotics}
\ccsdesc[100]{Networks~Network reliability}


% End generated code

\printccsdesc{}

\keywords{Linux; Software Architecture; Requirements Engineering}

\section{Introduction}

\section{Requirements Engineering}

\section{Software Architecture}

\section{Conclusions}

\bibliographystyle{abbrv}
\bibliography{sigproc}  % sigproc.bib is the name of the Bibliography in this case
%  and remember to run:
% latex bibtex latex latex
% to resolve all references

\balancecolumns{}
\end{document}
